\documentclass[a4paper, pt14]{article}

\usepackage[utf8]{inputenc}
\usepackage[T1]{fontenc}
\usepackage[slovene]{babel}
\usepackage{lmodern}
%\usepackage{hyperref}
\usepackage{amsmath}
\usepackage{amssymb}


\begin{document}

\title{%
Scheduling-Queues\\
  \large Finančni praktikum}
\author{Špela Bernardič, Ioann Stanković}
\date{10. \ 1. \ 2023}

\maketitle






\section{Navodilo naloge}
Naloga se ukvarja z algoritmi 'Scheduling-Queues', ki uporabljajo podatke napovedane s strojnim učenjem.
Potrebno je napisati simulacijo procesov, ki uporabi napovedane čase trajanja procesov in nato narediti analizo časa čakanja procesov.
Preveriti je potrebno, kako različene porazdelitve trajanja procesov in kvaliteta napovedi vplivata na povprečen čas čakanja v vrsti.
Napovedi časov trajanja procesov se lahko določi z dodajanjem šumov pravim vrednostim, lahko pa se uporabi naučen model.


\section{Opis problema}

S problemom razvrščanja opravil v vrsti se srečamo na različnih področjih. Za opravila nevemo nujno njihove dolžino trajanja, velikokrat pa jo laho ocenimo oziroma napovemo. 
Zato je poleg optimalnosti algoritmov, ki za razvrščanje uporabljajo dejanske čase trajanja opravila, smiselno analizirati tudi, primere, ko se uporabljajo napovedani časi trajanja opravila.
Opazovati je torej treba, povprečen čas čakanja opravila v vrsti. Želimo, da je ta čim krajši.

\section{Potek dela}
\subsection{Algoritmi razvrščanja procesov}

Za reševanje problema razvrščanja procesov v vrsti se uporabljaj različni algoritmi. V nalogi sva uporabila naslednje osnovne algoritme:
\begin{itemize}
  \item FCFS (First Come First Serve) je algoritem, ki izvaja procese po vrstnem redu njihovega prihoda.
  \item SJF (Non-Preemptive Shortest Job First) je algoritem, pri katerem se za naslednjo izvedbo izbere proces z najkrajšim časom trajanja. Ko se določi kateri proces se naj izvede naslednji, se ta izvede do konca. 
  \item PSJF (Preemptive Shortest Job First) je algoritem, pri katerem se proces z najkrajšim časom trajanja začne izvajati prvi. Ob prihodu novega procesa se le ta postavi v čakalno vrsto. Če pa pride proces s krajšim trajanjem od procesa, ki se trenutno izvaja, se trenutni proces ustavi in vrne v vrsto. Začne se izvajati proces s krajšim trajanjem.
  \item SRPT (Shortest remaining processing time) je algoritem podoben PSJF, vendar ta upošteva preostanke trajanj procesov.
\end{itemize}

Poleg osnovnih algoritmov sva napisala še variacje z napovedmi:

\begin{itemize}
  \item SPJF (Non-Preemptive Shortest Predicted Job First) za naslednjo izvedbo izbere proces z najkrajšim \textbf{napovedanim} časom trajanja. Ko se določi kateri proces se naj izvede naslednji, se ta izvede do konca. 
  \item PSPJF (Preemptive Shortest Predicted Job First) proces z najkrajšim \textbf{napovedanim} časom trajanja se začne izvajati prvi. Ob prihodu novega procesa se le ta postavi v čakalno vrsto. Če pride proces s krajšim \textbf{napovedanim} trajanjem od procesa, ki se trenutno izvaja, se trenutni proces ustavi in vrne v vrsto. Začne se izvajati proces s krajšim \textbf{napovedanim} trajanjem.
  \item SPRPT (Shortest Predicted remaining processing time) je algoritem podoben PSJF, vendar ta upošteva preostanek \textbf{napovedanega} trajanj procesov.
\end{itemize}



\subsection{Generiranje podatkov in predikcije}

V drugem delu naloge se bova ukvarjala s simulacijo podatkov. To bova naredila v programskem jeziku R, saj ta že vsebuje vgrajene funkcije za generiranje naključnih vrednosti.
Generirani podatki bodo morali vsebovati procese, njihove čase prihoda in koliko časa se (predvidoma) izvajajo.
Za simulacijo podatkov bova uporabila normalno in beta porazdelitev. 
Napovedi časa trajanja procesov pa bova določila tako, da bova prvotnim podatkom s spreminjanjem variance dodala šum. \\
%Za napoved bova v enem primeru vzela pričakovano vrednost, za dejanski čas pa simulirane podatke, poskusila bova tudi obratno.
Na simuliranih podatkih bova uporabila napisane algoritme iz prvega dela naloge in naredila analizo razultatov.

\section{Analiza}
\end{document}