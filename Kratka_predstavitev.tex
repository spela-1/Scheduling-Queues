
\documentclass[a4paper, pt14]{article}

\usepackage[utf8]{inputenc}
\usepackage[T1]{fontenc}
\usepackage[slovene]{babel}
\usepackage{lmodern}
%\usepackage{hyperref}
\usepackage{amsmath}
\usepackage{amssymb}

\begin{document}

\title{%
Scheduling-Queues\\
  \large Finančni praktikum}
\author{Špela Bernardič, Ioann Stanković}
\date{16. \ 12. \ 2022}

\maketitle

\section{Navodilo naloge}
Naloga se ukvarja z algoritmi 'Scheduling-Queues', ki uporabljajo podatke napovedane s strojnim učenje.
Potrebno je napisati simulacijo vrste, ki uporabi napovedane čase trajanja opravil in jo nato uporabiti za analizo.
Preveriti je potrebno kako različe porazdelitve trajanj opravila in kvaliteta napovedi vplivata na povprečen čas čakanja v vrsti.
Napovedi časov trajanja opravil se lahko določi s dodajanjem šumov pravim vrednostim, lahko pa se uporabi naučen model


\section{Obrazložitev algoritmov}

\section{Potek dela}

Najprej se bova lotila pisanja algoritmov FCFS, SJF, PSJF, SRPT  in pa variacije z napovedmi torej SPJF, PSPJF, SPRPT. To bova naredila v programskem jeziku Python.
Vsi napisani algoritmi bodo vračali koliko časa v povprečju opravila čakajo v vrsti.\\
Drugi del naloge je simulacija podatkov. Ta del bova naredila v programskem jeziku R, saj ta že vsebuje vgrajene funkcije za generiranje naključnih vrednosti.
Generirani podatki bodo morali vsebovati opravila, njihove čase prihoda in koliko časa se (predvidoma) izvajajo.
Napovedi Časa izvajanja bova določila tako, da bova prvotnim podatkom dodala šum.
Na simuliranih podatkih bova uporabila napisane algoritme iz prvega dela naloge in naredila analizo razultatov.
\end{document}




